\newcommand{\defbox}[2]{
\vspace{1cm}
\noindent
  \begin{tcolorbox}[colframe=red, colback=white, width=\linewidth, title=#1]
    #2
  \end{tcolorbox}
  \vspace{1cm}
}

\newcommand{\proofbox}[3]{%
    \begin{proof}
        \\ \textbf{Annahme:}
        \begin{align*}
            #1
        \end{align*}
        \textbf{Zeige:} #2
        \begin{align*}
            #3
        \end{align*}
    \end{proof}%
}


\section{Beweise}
\rule{\textwidth}{0.4pt}
\subsection{Statements} Ein \textbf{Statement/ Aussage} ist ein Mathematischer Ausdruck der entweder wahr oder Fallsch ist. 
\subsubsection*{Beispiel: } 
\begin{itemize}
    \item $2 \in \{x \in \mathbb{R} \vert x < 5\}$ (wahr)
    \item $3^2 + 5^2 = 8^2$ (falsch)
\end{itemize} 
Dabei werden Ausdrüche wie $0 < x < 1$ verwendet um Mengen zu definieren. \[A = \{x\in\mathbb{R} \vert 0 < x < 1\}\] Wichtig ist hierbei der \textbf{Wahrheitswert} eines offenen Ausdrucks $0 < x < 1$ hängt von gewähltem x ab. Also ist 
\begin{itemize}
    \item $x = 1/2$ (wahr) 
    \item $x = 5 $ (falsch)
\end{itemize}
Die \textbf{Domäne} ist hierfür wichtig zu beachten. Für $\mathbb{N}$, gibt es kein $x$ s.t. $0 < x < 1$, aber es gibt welche für $\mathbb{R}$

\subsection{Formal mathematical proofs} Ein \textbf{formaler mathematischer Beweise} besteht aus einer nummerierten Sequent von \textbf{wahren Aussagen}. Jede Aussage in einem Beweis ist ein \textbf{Annahme} oder \textbf{folgt} aus vorherigen Aussagen dirch Ableitungsregel/ Inferenzregel (rule of inference). Die Letzte Aussage ist die die wir bewiesen haben.\\
$\Rightarrow$ Offene Aussagen können in Beweisen nicht auftreten!

\defbox
{
    Beispiel einer Inferenzregel: set definition rule
}
{
    Wenn ein Element in einer Menge ist, dann können wir definierende Eienschaften ableiten. Andererseits, wenn es die definierende Eigenschaften erfüll, dann können wir ableiten das dasd Element in der Menge ist.
}

\subsection{Set definition rule: Beispiel} Definiere $C = \{x \in \mathbb{R}\vert x < 2\}$ \\ ($x < 2 \land x \in \mathbb{R}$ ist die definierende Eigenschaft). Dabei gibt es zeiw Möglichkeiten für die Ableitung 
\begin{itemize}
    \item [Möglichkeit 1] 
    \begin{enumerate}
        \item $a \in C$
        \item $a < 2 \land a \in \mathbb{R} (1; def C)$
    \end{enumerate}
    \item [Möglichkeit 2] 
    \begin{enumerate}
        \item $b < 2 \land b \in \mathbb{R} $
        \item $b \in C (1; def C)$
    \end{enumerate}
\end{itemize}
Jede Aussage in dem Bewies hat eine Nummer. Wir begründen wie wir eine Aussage ableiten, zb $(1; def C)$ bedeutet wir leiten die aktuelle Aussage aus Aussage 1 mit der Definition von $C$ und der set definition rule ab.

\subsubsection*{Bemerkung: $\land b \in R$} wird oft ausgelassen, wenn Kontext es zulässt.

\subsection{Macro-steps in proofs}
\subsubsection*{Problem: } Schauen wir uns folgenden Bewwis an: \\ (ass = Annahme (assumption) und prop = Eigenschaft(property))

%\proofbox
%{
%    1. & X = \{x\in\mathbb{R} \,|\, x<1\} \\
%    2. & a\in X
%}
%{    
%    $a<2$
%}
%{
%    1. &\quad a\in X                   && \text%{(ass 2)} \\
%    2. &\quad a<1                      && \text{(1, ass 1; def X)} \\
%    3. &\quad 1<2                      && \text{(prop R)} \\
%    4. &\quad a<2                      && \text{(2, 3; prop R)}
%}
    

\textbf{Ist das ein akzrptabler Beweis?} Akzeptanz von Makro-Schritten wie "\textbf{prop $\mathbb{R}
$}" hängt von der Zielgruppe ab! \\ Welche Eigenschaft von $\mathbb{R}$ wurde benutzt?

\subsection{Einfache Beweistechniken}
\subsubsection{Beweis durch beispiel}
\textbf{Beispiel: } Zeigen Sie es gibt einePrimzahl zwischen 80 und 90.\\ \textbf{Idee: } Zeugen angeben für Primzahl $(p)$ für die die Aussage gilt.
\begin{proof}
    Wähle $p = 83$
\end{proof} 
Ist das ausreichend?\\Eigenflich, \textbf{NEIN}. Wie müssen noch \textbf{zeigen} das 83 tatsächlich eine Primzahl ist. \\Das können wir tun in dem wir alle Teiler ausprobieren. 

\subsubsection{Wiederlegen von Behauptungen} 
\textbf{Behauptung: } Nimm an $n$ ist eine Primzahl größer als 1. Dann ist $2^n - 1$ ebenfalls eine Primzahl. \\Können Sie die Behauptung beweisen? Try hard $\cdots$\\ Wenn Sie es nicht können, dann sollen Sie darüber nachednken die Behauptung zu wiederlegen. \textbf{Eine Primzahl $n$ für die $2^n - 1$ nicht prim ist, ist genug!}\\ Das Gegenbeispiel ist $n = 11$ da $11$ prim ist aber \[2^{11} - 1 = 2047 = 23 \cdot 89\] keine Primzahl ist!

\subsection{$\forall$ Aussagen beweisen}
\subsubsection*{Inferenzregel für definierte Beziehungen}

\defbox
{
    The definition rule 
}
{
    Angenommen, es wurde eine Beziehung definiert. Wenn die Beziehung gilt (in irgendweinem Beweisschritt oder Annahme), dann kann die definierende Eigenschaft abgeleitet werden. Andererseits, wenn die definierende Eigenschaft gilt, dann kann die Beziehung abgeleitet werden.
}

\textbf{Beispiel: } Für Mengen $A$ und $B$, definiere $A$ ist \textbf{Teilmenge} von $B$, $A \subseteq B$, wenn \textbf{für alle $x$ mit $x \in A : x \in B$}. Mit anderen Worten: \\ $A \subseteq B$ if and only if (iff) $\forall x ((x \in A) \rightarrow (x \in B))$ ist wahr.
\begin{itemize}
    \item [Möglichkeit 1:]
        \begin{align*}
            1. &\quad A \subseteq B                   && \text{(ass 2)} \\
            2. &\quad \text{für alle x s.t.} x \in A : x \in B                      && \text{(1, def $\subseteq$)} \\
        \end{align*}
    \item [Möglichkeit 2:]
        \begin{align*}
            1. &\quad \text{für alle x s.t.} x \in A : x \in B                      && \text{(1, def $\subseteq$)} \\
            2. &\quad A \subseteq B                   && \text{(ass 2)} \\
        \end{align*}
\end{itemize}

\subsubsection*{Inferenzeregel für $\forall$}
Sei $\mathfrak{P} $ eine Formel. Beispielsweise steht $\mathfrak{P}(x)$ für $x \in A$ und $\mathfrak{Q}(x)$ steht für $x \in B$. Dann kann "\textbf{für alle x s.t. $x \in A: x \in B$}" als "\textbf{für alle x s.t. $\mathfrak{P}(x) : \mathfrak{Q}(x)$}" geschrieben werden.

\defbox{Regeln um $\forall$ Aussagen zu beweisen (pr $\forall$)}{ Um Aussagen der Form "\textbf{für alle $x$ in s.t. $\mathfrak{P}(x) : \mathfrak{Q}(x)$}", zu beweisen nimmt man an $x$ sei \textbf{beliebig gewähtes Element (eigenvariable)} s.t $\mathfrak{P}(x)$ wahr ist. Dann zeige man $\mathfrak{Q}(x)$ ist wahr.}
\textbf{Generaliesirungen} z.B. "für alle $x, y$ s.t. $\mathfrak{P}(x,y):\mathfrak{Q}(x,y)$" möglich

\subsubsection{Ein Beispiel} Sei $C = \{x \in \mathbb{R} \vert x < 1\}$ und $D = \{x \in \mathbb{R}\vert x < 2\}$. Zeige $C \subseteq D$! 

%\proofbox
%{
%    1. & C = \{X \in \mathbb{R} \vert x < 1\} \\
%    2. & D = \{x \in \mathbb{R} \vert x < 2\}
%}
%{
%    \ensuremath{C \subseteq D}
%}
%{
%    1. &\quad\quad \text{Sei }x \in C \text{ beliebig } && \text{ } \\
%    2. &\quad\quad x < 1 && \text{(1, ass 1; def C)} \\
%    3. &\quad\quad x < 2 && (2, \text {prop} \mathbb{R} ) \\
%    4. &\quad\quad  x \in D && \text{(3, ass 2; def D)} \\
%    5. & \quad \text{für alle } x \in X : x \in D && (1 - 4; \text{pr} \forall) \\
%    6. & \quad C \subseteq D && (5, \text{def} \subseteq)
%}


\textbf{Wie können wir "für alle $x$ s.t $\mathfrak{P}(x) : \mathfrak{Q}(x)$" wiederlegen?}

\subsubsection{Bemerkungen}
\begin{itemize}
    \item Durch Einrückung kennzeochnen wir \textbf{Teilbeweise} die von einer ANnahme wie "Sei $x \in C$ beliebig" abhängen.
    \item Eine Annahme hat keine Begründung.
    \item Teilbeweise 2-4 basieren auf der Annahme in 1
    \item Schritte aus 1-4 können nicht in Begründungen auftauchen, sobald der Teilbeweis fertig ist (d. h, \textbf{nach pr $\forall$ in 5})
    \item Wir schreiben oft "für alle $x \in C : x \in D$" statt "für alle $x s.t. x \in C : x \in D$" 
\end{itemize}

\section{Verwenden von $\forall$ Aussagen}
\subsection{Inferenzeregel um $\forall$ Aussagen zu verwenden}
\defbox
{
    Die Regel um $\forall$ Aussagen in Beweisen zu verwenden (us $\forall$)
    }
{
    Wenn wir wissen das eine Aussage "für alle $x$ s.t $\mathcal{P}(x) : \mathcal{Q}(x)$" wahr ist und wir $\mathcal{P}(t)$ berets als einen Schritt für eine Variable $t$ im Beweis haben, dann können wir $\mathcal{Q}(t)$ ableiten.
}

\subsubsection{Beispiel}

    \begin{align*}
        1. &\quad t \in A                   && \\
        2. &\quad \text{für alle x s.t.} x \in A : x \in B                      && \\
        3. &\quad t \in B && (1, 2; us \forall)\\
    \end{align*}    

\subsubsection{Beispiel}

\begin{align*}
    1. &\quad |a| < |b|                   && \\
    2. &\quad \text{für alle x s.t.} |x| \leq |y| : x^2 \leq y^2                      && \\
    3. &\quad a^2 < b^2 && (1, 2; us \forall)\\
\end{align*}    

\subsection{Beispiel}
Seiesn $A, B, C$ Mengen. Zeige $\subseteq$ ist transitiv, d. h, zeige $A \subseteq B$ und $B \subseteq B$, dann $A \subseteq C$.


